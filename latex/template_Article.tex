\documentclass[]{article}
\usepackage[margin=1in]{geometry}
\usepackage{graphicx}
\usepackage{hyperref}
\hypersetup{
	colorlinks=true,
	linkcolor=blue,
	filecolor=magenta,      
	urlcolor=cyan,
}
\graphicspath{ {images/} }
\setlength{\parindent}{0pt}

%opening
\title{Data Standardization and Exchange in Macromolecule Crystallography}
\author{Ethan Holleman}

\begin{document}

\maketitle

\begin{abstract}
	I spent the summer of 2020 developing Polo: A graphical user interface for high-throughput crystallography, the vast majority of work for which was done over a period of about 12 weeks. Over that short time I encountered a number of problems but one that stood out to me was the lack of consensus regarding digital representations of crystallization experiments and outcomes. Here, I express a few of my thoughts on the topic.

\end{abstract}

\section{Where is Crystalization's pdf File?}

This passed summer, I spent much of my quarantine time working as a summer intern in the BioXFEL internship program. My goal was to build a side project I had been working on previously, a graphical user interface integrating automated image classification algorithms, to fruition. If you are interested in seeing the results you can download the program from the \href{https://github.com/Hauptman-Woodward/Marco_Polo}{GitHub page}. I got to work under the supervision of Dr. Sarah Bowman, the director of the high-throughput crystallography center at Haputman-Woodward Medical Research Institute. 

One of the first challenges I faced was just writing code that was robust enough to parse all the different file formats that translate what the lab was doing day-to-day for its users to a digital record. This included csv files of chemical data, xml files describing screens, and of course image files. Behind the schemes of Polo there are hundreds of lines of code that gather this information, connect it, and then present it to the user in a way that makes it easy to assume everything is tightly woven together when the bit-level reality is that this is mostly an illusion. An organizational apparition that evaporates into the RAM ether once you close the program. 

To make my own life easier and allow Polo users to save their work in an exchangeable way I began looking for accepted file standards used in the crystallization side of X-ray crystallography but didn't find much outside of what the lab was already going, at least not anything that brought everything I wanted together into the same place. I specially wanted a format that would 

\begin{itemize}
	\item Directly associate image data to classifications and other metadata
	\item Describe chemical information in an easily parasable format
	\item Reflect the design of a typical crystallization experiment
	\item Be implementable into my program in a week or less
\end{itemize}

The last item is not longer a constraint but I included it since it weighed significantly the direction I actually went in. The best solution I found, and arguably the best solution currently exiting was the json formatted API responses from HWI's own \href{http://xtuition.org/}{Xtution database}. While Xtution is a tremendous resource, I needed a slightly more integrated solution and ended up putting together (some would call hacking) a custom json format that is precipitated directly from the data structures Polo holds in memory. This had the advantages of working and being fast to implement (see item 4 above) but created a file that is not easily generalization beyond its use in my own program.

My thought during and increasingly so now as the time crunch of my ten-week internship is no longer a driving factor of my decision making, is where is the crystallization pdf? The format that I can just open up and know what I am going to expect without having to read documentation or do any guesswork? The diffraction side of crystallography seems to be way ahead and has been for a long time.

\begin{itemize}
	\item Protein Data Bank format (.pdb), introduced 1977 \cite{pmid875032}
	\item Collaborative Computational Project Number 4 format (.ccp4), introduced 1979
	\item Crystallographic Information File (.cif), introduced 1991 \cite{Hall:es0164}
\end{itemize}

My theory is that traditionally, crystallization experiments are small and infrequent enough where you can generally get away using a spreadsheet or just making a table in your lab notebook. The problems with these methods are increasingly revealed as scale and frequency increase, \textit{i.e} the high-throughput crystallography setting. However the relative small-scale of this data does not confine the problems it creates to this same domain. The main problem I see that is created by a lack of standardization is that it forces the development of custom "in-house" solutions (to which to some degree, Polo is apart of). Researchers not seeing an obvious solution follow what they believe to make the most sense at the time and for their laboratory. This could mean tables in Excel, slides in PowerPoint, pictures in a lab notebook or drawings done in MS Paint. This makes it difficult to share, validate, and as time passes even locate (as anyone who has searched for a years-old word file knows). 

Before the French Revolution, there were an estimated 250,000 different units of weight and measure in use. From my limited viewpoint, I think this may be an apt, although slightly exaggerated metaphor for the current state of crystallization experiment data representation. Adoption of one, or a small collection of generally agreed-upon and well documented file formats would allow for the easier development of universally accessible software suites that would in turn make managing, tracking and sharing of crystallization data consistent and trivial.

\section{Towards Standardization}

\subsection{Looking Back}

\subsection{Approaches}


\bibliographystyle{plain}
\bibliography{reference}

\end{document}
